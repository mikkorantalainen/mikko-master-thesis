\subsection{Vihko (kehitysnimi, 2003-2004, k�ytt��notto syksyll� 2004)}

Peda.net Vihko on kirjoittajan nykyinen kehitysprojekti joka tarjoaa yksitt�iselle koululle riitt�v�n oppimisymp�rist�n. T�m� sovellus ei tue kurssien j�rjest�mist� usean eri oppilaitoksen kesken vaan on suunniteltu yksitt�isen koulun tarpeita varten. Yll�pit�j�n kannalta merkitt�vi� ominaisuuksia ovat k�ytt�jien hallinta monitasoisten ryhmien kautta, kurssien ja kurssimateriaalien oikeuksien m��rittely luku- ja kirjoitusoikeuksien avulla ja versionhallinta sek� kaiken materiaalin vapaa linkitt�minen mihin tahansa j�rjestelm�n sis�ll�. Loogisella tasolla vihkossa yll�pidet��n kaksisuuntaista hierarkiaa, joka muodostuu erilaisista dokumenttityypeist� tai moduuleista: kansioista, tiedostoista, keskusteluista ja ty�tiloista. Jokainen moduuli voi sis�lt�� muita moduuleita ja jokainen moduuli voi n�ky� monen eri moduulin sis�ll�. Tietokanta mahdollistaa jopa moduulin n�kymisen itsens� sis�ll�. Osaa toiminnoista on rajoitettu k�ytt�j�n toimien helpottamiseksi.

\termlist

\begin{terms}

\item[HTML] HyperText Markup Language (\alt{HTML}) on tekstimuotoinen esitystapa rakenteisen dokumentin esitt�miseen.

\item[XHTML] Extensible HyperText Markup Language (\alt{XHTML}) on HTML-kielen XML-kielinen versio, jolla kirjoitetuissa dokumenteissa voidaan k�ytt�� my�s muita XML-kielen sovelluksia, kuten esimerkiksi MathML-kielt� matemaattisten merkint�jen tekemiseen.

\item[ohjelman tila] on tietokoneohjelman se osa, joka sis�lt�� kaiken ohjelman tarvitseman tiedon. Esimerkiksi tekstink�sittelyohjelman tapauksessa tila sis�lt�isi muokattavan tekstin lis�ksi mm. kohdistimen sijainnin ja ty�kalupalkkien sijaintitiedon.

\item[tapahtuma] on ohjelmoinnissa k�ytetty malli, jossa ohjelman tilaa ohjaillaan tilaa muuttavilla tapahtumilla. Esimerkiksi ''nappula $A$ painettiin alas'' voisi olla tapahtuma. Tapahtumien vahvuutena on, ett� sama tapahtuma voidaan tuottaa monessa eri paikassa, mutta tapahtuma tarvitsee k�sitell� vain yhdess� paikassa -- kun k�sittelyyn tarvitaan muutoksia, tarvitsee muutos tehd� vain yhdess� paikassa.

\item[skriptikieli] vastaa ohjelmointikielt�, mutta puhuttaessa skriptikielest� tarkoitetaan yleens� kielt�, joka on liitetty varsinaisen sis�ll�n yhteyteen. Usein skriptikielen ilmaisuvoima on pienempi kuin t�yden ohjelmointikielen -- esimerkiksi muiden tiedostojen avaaminen tai muokkaaminen ei ole mahdollista. 

\item[]

\item[]

\end{terms}
